\documentclass[10pt, twoside]{article}

\usepackage{style}

\title{\textbf{\huge Incremental Computation and Visualisation\\of Three-Dimensional Pareto Frontiers}}
\author{Gustavo Martins\\\normalsize \textit{University of Coimbra, Portugal}}
\date{October, 2016}

\begin{document}
	
\maketitle

\begin{abstract}
Multi-objective optimisation has been extensively studied and used for a long time, having practical applications in almost every field there is, from Economics and Finance to Logistics and Engineering, among many others. As result, many methods, techniques and algorithms have been developed over the years in order to solve the most diverse problems, such as resource management, network and systems design, or ..., to name a few. However, there is a lack of tools that allow a graphical visualisation of the solutions produced by these methods, especially for problems with three or more objectives. In this work, a method to compute the facets of a three-dimensional Pareto frontier, which is based on a well-known algorithm for finding the optimal solutions of a set, is proposed. Additionally, an incremental approach to the frontier computation, allowing for a reasonably fast update of the frontier, is presented. Further applications for this method are also explored.\\

\noindent\textbf{Keywords:} Computational Geometry; Multi-Objective Optimisation; Pareto Frontier; Data Visualisation.
\end{abstract}

\section{Introduction}

assumir o resto do relatorio como minimização nas 3 dimensões
\section{Find the optimal solutions of a set}

\paragraph{Algorithm} One of the most well-known algorithms to find the set of optimal solutions was first proposed by Kung in 1975 \cite{kung_optima}, and even though it was presented as an algorithm to find the maxima of a set, it can be applied to find both the maxima and minima by using the definition of dominance. The methodology is to divide the problem into several sub-problems in the lower dimensional space --- allowing the algorithm to be applied to any number of dimensions. The algorithm achieves that by pre-sorting the solutions by one of the dimensions. That way, a solution is optimal only if its projection onto the other dimensions is also optimal --- given that the previous solutions in the sorting list dominate the current solution in the dimension by which the sorting is performed.

\begin{algorithm}[h]
	\begin{algorithmic}[0]
		\Statex
		\State\textbf{\hspace{-7pt}Input:} $S_i$ - set of solutions in $\mathbb{R}^d$; $d$ - number of dimensions (objectives)
		\State\textbf{\hspace{-7pt}Output:} $S_o$ - set of optimal solutions of $S_i$
	\end{algorithmic}
	\begin{algorithmic}[1]
		\vspace{3pt}
		\Function{optima}{$S_i$, $d$}
		\State $Q \leftarrow$ \Call{sort}{$S_i$, $d$} \Comment{Queue containing $S$ sorted by coordinate $d$}
		\State $S_o \leftarrow \emptyset$
		\State $S^\ast \leftarrow \emptyset$ \Comment{Set of optimal solutions, of dimension $d-1$}
		\While{$Q \neq \emptyset$}
		\State $s \leftarrow$ $Q$.\Call{pop}{\null} \Comment{Extract element from the queue}
		\State $s^\ast \leftarrow$ \Call{project}{$s$, $d-1$} \Comment{Projection of $s$ onto the first $d-1$ dimensions}
		\If{$s^\ast \nprec S^\ast$}
		\State $S_o \leftarrow S_o \cup \{s\}$
		\State $S^\ast \leftarrow$ \Call{optima}{$S^\ast \cup \{s^\ast\}$, $d-1$}
		\EndIf
		\EndWhile
		\State \Return $S_o$
		\EndFunction
	\end{algorithmic}
	\caption{Optimal solution set \cite{kung_optima, preparata_comp_geom, fonseca_eaf}}
	\label{alg:optima}
	\algrule[1pt]
	\small When $d$ reaches the value 1, there is no need to perform lines 3--10. As replacement, the following suffices: $S_o \leftarrow \{Q.\textsc{pop}(\null)\}$.
\end{algorithm}

\paragraph{Three-dimensional implementation} The first step of the algorithm consists of sorting the solutions by the following order: $\{x_1,x_2,x_3\}$ --- any sorting algorithm with logarithmic time complexity is a valid choice. Given the order, the solutions projection contains the second and third dimensions --- $x_2$ and $x_3$. The more complex operations are the optimality verification and the update of the lower dimensional space --- lines 8 and 10 of Algorithm \ref{alg:optima}. This can be achieved by using a balanced binary tree, called \textit{projection tree}, that stores its elements in the following order: $\{x_2,x_3\}$. That way, verifying if a solution is optimal or not consists of inserting the solution in the projection tree and comparing it with its predecessor in the tree. Since the predecessor was inserted first in the projection tree and the fact that it is the predecessor implies that it has an equal or lower value of $x_1$ and $x_2$, respectively, than the current solution. Thus, the current solution is optimal only if it has a lower $x_3$ value than its predecessor. In the case that the solution is optimal, the remaining operation is to update the projection tree, which can be done by iterating through the successors in the tree, removing them in the process, until one that has a lower $x_3$ value is reached.


\paragraph{Time complexity} Given the initial sorting of the solutions by a particular dimension, the algorithm starts with $\mathcal{O}(n\log n)$ time complexity. Also, each point is inserted in the projection tree even before verifying its optimality, which adds another $\mathcal{O}(n\log n)$ to the overall complexity. Verifying a solution's optimality implies accessing the predecessor in the projection tree, adding $\mathcal{O}(n)$ --- given that the binary tree is threaded; otherwise, $\mathcal{O}(n\log n)$. Finally, if the solution is not optimal, it must be removed from the tree. Otherwise, the now dominated projected solutions must be removed from the tree --- these are the successors, accessible in $\mathcal{O}(1)$ time each. However, it is known that a solution is inserted only once in the tree, and therefore, removed at most once, either in the first or in the second scenario --- it can stay in the projection tree until the end, and not be removed. Therefore, the complexity increases another $\mathcal{O}(n\log n)$, totalling a complexity of $\mathcal{O}(3\times n\log n + n) = \underline{\smash{\mathcal{O}(n\log n)}}$.

\paragraph{Space complexity} Beyond the set that stores the solutions, the only auxiliary storage necessary is the projection tree, containing at most the total number of solutions in the set --- in the case they are all optimal. Therefore, the space complexity of the algorithm is $\underline{\smash{\mathcal{O}(n)}}$. 
\section{Compute the Pareto frontier}

save
\section{Incremental approach}

\paragraph{Methodology} The objective of an incremental approach to the frontier computation is to be able to perform modifications --- adding or removing points --- without having to recalculate every facet. The implementation consists of knowing whether to (re)compute the facets or to just update the projection tree --- much like the optima algorithm --- for each point of the set for each dimensional sweep. Also, whenever the points that have been scheduled for addition and removal are finished processing, the algorithm can be terminated, since the rest of the facets will remain intact.
Another optimisation to be considered is whenever the points scheduled for removal are at the end of the sorting list for a particular dimension. If this is the case, it would be enough to remove the facets generated by these points and translate the vertices from the old limits of the frontier to the new limits. Likewise, whenever the frontier's span increases --- addition of points beyond the limits ---, the vertices at the limits can be translated to the new limits, possibly avoiding some facet recalculation.

\paragraph{Points scheduled for addition and removal} Between frontier computations, any number of points can be scheduled for addition or removal. The incremental approach recalculates facets only when these points influence the computation, i.e.\ when they are present in the projection tree. When iterating through the points' predecessors and successors in the projection tree, the points scheduled for addition are treated as regular points, while the ones scheduled for removal are ignored. The latter are still verified and removed from the projection tree when dominated.

\paragraph{Advantages and disadvantages} The main \textit{advantage} of an incremental approach is that not every facet needs to be recalculated, reducing not only processing time but also memory management. Still, the projection tree needs to be updated. In the worst case scenario, one has to process every single point, inserting them in the projection tree, only to compute the facets generated by a single point at the end. However, it is still a better choice than to recalculate the whole frontier from scratch. The main \textit{disadvantage} is memory consumption. In order to efficiently add and remove points from the frontier, it is advisable to store the set of points in a order-based structure, like a binary tree, so to avoid sorting the entire set whenever the frontier is to be updated. Note that three structures are necessary, one for each dimension. Furthermore, if one desires to perform the optimisations described at the end of the paragraph ``Methodology'', the same needs to be done for the vertices, so that those at the limits are easily accessible.

\paragraph{Algorithm conditions:}\ \\[2pt]The dimensional sweep can be terminated when none of the following conditions proves true:
\begin{itemize}
\item Projection tree contains points scheduled for addition and/or removal
\item Points scheduled to be added and/or removed are still to be processed
\item Facet sharing (vertices have been saved)
\item Facet sharing break
\end{itemize}
When deciding whether to (re)compute the facet (1) or to just insert the point in the projection tree (2), the following verifications must be performed (in this order):
\begin{itemize}
\item Point is scheduled for removal $\rightarrow$ (2)
\item Point is scheduled for addition, or ...
\\[2pt] Projection tree contains points scheduled for addition and/or removal, or ...
\\[2pt] Point has the same sweep dimension value as the next point scheduled to be added or removed, or ...
\\[2pt] Facet sharing (vertices have been saved), or ...
\\[2pt] Facet sharing break $\rightarrow$ (1)
\item Otherwise $\rightarrow$ (2)
\end{itemize}

\paragraph{Special case: facet sharing break} Due to the incremental nature of the method, a new special case must be considered --- mentioned above as \textit{facet sharing break}. This occurs when the projection of a new point $c$ dominates or is equal to the intersection of two already existing points, $p$ and $q$, that share a facet. When this happens, the facets of $p$ and $q$ need to be recalculated. However, since they no longer share the same facet and $c$ is partially dominated by $p$, and therefore removed from the projection tree, there would be no reason to recompute the facets of $q$. To solve this problem, a verification at the end of the facet computation routine is necessary, activating a flag variable to force the recalculation of the facets of $q$. The verification --- placed between lines 33 and 34 of Algorithm \ref{alg:facet_computation} --- is as follows:
\begin{align*}
c.\textsc{state}() = 1\ \textbf{and}\ q_1 = p_1\ \textbf{and}\ p.\textsc{state}() = 0\ \textbf{and}\ q.\textsc{state}() = 0
\end{align*}
where \textsc{state}() returns $1$ if the respective point is scheduled for addition, $-1$ if it is scheduled for removal, and $0$ otherwise.

\paragraph{Complexity} Neither the time nor the space complexities of the algorithm are changed by the incremental approach. The only thing to note is the increase in memory usage, already described in the paragraph ``Advantages and disadvantages''.
\section{Conclusions and Further developments}

\ \todo[inline]{TODO: Practical application to the AFT \cite{fonseca_eaf, ibanez_eaf}}\ 
\bibliography{references}

\end{document}